\documentclass{article}
\usepackage[utf8]{inputenc}
\usepackage{amsmath}
\title{0D numerical solution}
\author{andreapiccolo89 }
\date{September 2022}
\usepackage{mathtools}
\begin{document}

\maketitle

\section{Introduction}

I take the work done during the coding friday and expanding a bit.

\begin{itemize}
    \item Setting up the equations starting from the definition of necking; 
    \begin{enumerate}
        \item Rheological consideration;
        \item Introduction of the main Forces ($F_B$ and $F_D$)
        \item Definition of $\tau_{B,0}$ and $\tau$; 
        \item Definition of the dimensional equation to solve, with the characteristic scaling
    \end{enumerate}
    
    \item Definition of the characteristic dimensions and additional relation
    
    \item Derivation of the $\tau$
    \begin{enumerate}
        \item Rearranging the equation associated with the force balance;
        \item Introduction of the following adimensional number: $\Lambda$,$\Psi$ and $\gamma$
        \item Definition of $\tau$ as  a function of $\frac{dD}{dt}$,$\tau_B$ and $\Lambda$,$\Psi$ and $\gamma$. 
    \end{enumerate}
    \item Final equation and introduction of the therminology
\end{itemize}


\section{Derivation 0D numerical equations}
\subsection{Main reference equation and introductory definitions}
We have:
\begin{equation}
    \dot{\varepsilon} = -\frac{1}{D} \frac{dD}{dt}
    \label{eq:eq1}
\end{equation}
and
\begin{equation}
    \dot{\varepsilon} = B_{n} \tau^n + B_{d} \tau
    \label{eq:rheology0}
\end{equation}

this can be expressed as:
\begin{equation}
    \dot{\varepsilon} = B_{n} \tau^n \left[1 + \frac{B_{d}}{B_{n}} \tau^{1-n}\right]
    \label{eq:rheology1}
\end{equation}

transition stress:
\begin{equation}
      \tau_t^{n-1} = \frac{B_{d}}{B_{n}}
\end{equation}
Put that in the equation above:
\begin{equation}
    \dot{\varepsilon} = B_{n} \tau^n \left[1 + \left(\frac{\tau}{\tau_t}\right)^{1-n}\right]
    \label{eq:rheology2}
\end{equation}

Now, I would like to add a further definition. In my numerical approach I imposed that $\eta_d$ is $\eta_d = \Xi \eta_n$ at $\tau_{B}$ (the buoyancy stress imposed by the stalled slab). Where $\Xi$ is the viscosity contrast between the two mechanisms at the reference condition (i.e. $\tau_B$). 

From this reference condition, I can derived $B_{n|d}$. Therefore, this strategy allows me to do the following: 

\begin{align}
B_d &= \frac{1}{2\eta_n\Xi} \\
B_n &= \frac{\tau_B^{1-n}}{2\eta_n\Xi}    
\end{align}

then: 
\begin{equation}
    \tau_t^{n-1} = \frac{\frac{1}{2\eta_n\Xi}}{\frac{\tau_B^{1-n}}{2\eta_n}}
    \label{eq:transition_stress0}
\end{equation}

doing some arrangments: 

\begin{equation}
     \sqrt[n-1]{\tau_t^{n-1}} =  \sqrt[n-1]{\frac{\tau_B^{n-1}}{\Xi}}
    \label{eq:transition_stress1}
\end{equation}
and by conveniently introducing $\xi = \Xi^\frac{1}{1-n}$

\begin{equation}
    \tau_t = \xi\tau_B
    \label{eq:transition_stress2}
\end{equation}


So, let's come back to the equation 1 and rearrange it to describe the problem that I want to solve: 

\begin{equation}
    \frac{dD}{dt} = -D \left\{ B_n \tau^n \left[ 1+ \left( \frac{\tau}{\xi \tau_B} \right)^{1-n} \right] \right\}
    \label{eq:equation_neck0}
\end{equation}

The stress in the necking region is given by:
\begin{equation}
    \tau = \frac{1}{2}\frac{F_B+F_D}{D}
\label{eq:effective_stress0}
\end{equation}

where: 
\begin{align}
    F_B &= \Delta \varrho g L_0 D_0 \\ 
    F_D &= -2\frac{dD}{dt}\eta^{UM}_{eff,0}\left(\frac{D_0}{D}\right)^2\frac{L_0 \alpha}{s}
    \label{eq:Drag_Force}
\end{align}

The equation from which I would like to start the derivation is the following

\begin{equation}
    0 = -D^{a} l_c \left\{ (B_n^{a} \tau_c^{-n}{t_c}^{-1})  \left(\tau^{a}\tau_c\right)^n \left[ 1+ \left( \frac{\left(\tau^{a}\tau_c\right)}{\xi (\tau^{a}_{B,0}\tau_c)} \right)^{1-n} \right] \right\} -\left(\frac{dD^{a}}{dt^{a}}\frac{l_c}{t_c}\right)
  \label{eq:Main_equation}
\end{equation}

Where $\tau^{a}= \tau/\tau_c$ represents the not dimensional effective stress. While $D^{a}$ represents the not dimensional thickness. 
\subsection{Characteristic value and additional relations}
Now let's start defining additional and important relation and what I believe are the most important characteristic length: 
\begin{align}
    l_c &= D_0 \\ 
    \tau_c &= \tau_{B,0}\\
    &= \frac{F_B}{2D_0} \\ 
    &= \frac{\Delta \varrho g L_0 D_0}{2 D_0} \\ 
    &= \frac{\Delta \varrho g L_0}{2}\\
    t_c &= \frac{1}{\dot{\varepsilon}_c} \\
    \dot{\varepsilon}_c &= \left\{ B_n \tau_c^n \left[ 1+ \left( \frac{\tau_c}{\xi \tau_B} \right)^{1-n} \right] \right\}\\
     \frac{dD}{dt}_{c} &= l_c \dot{\varepsilon}_c\\
     &= \frac{l_c}{t_c}
\end{align}

Additionally, it can be possible define an other quantity $\eta^S_{eff,0}$. This quantity is the effective viscosity of the slab at given reference condition (i.e. $\tau=\tau_{B,0}=\tau_c$). This quantity allows to define the $\tau_{B,0}$ in this alternative manner:

\begin{equation}
    \tau_c = \tau_{B,0} = 2\eta^S_{eff,0}\dot{\varepsilon}_c
    \label{eq:tau_B_0_eq0}
\end{equation}



Then, we can tackle the problem represent by $\tau$ and introducing other useful relation for the derivation: 

\begin{align}
    \tau_B &= \frac{\tau_{B,0}D_0}{D} [Pa] \\
    \tau_B^{a}&= \frac{D_0}{D} 
\end{align}
\subsection{Derivation of the effective stress}
The basic equation are set up. In the following part I derive a definition for $\tau$, the effective stress \eqref{eq:effective_stress0}, using the relations in \eqref{eq:Drag_Force}.

\begin{align}
\tau &= \frac{F_B}{2D}\left(1+\frac{F_D}{F_B}\right)\\
&= \frac{\tau_{B,0} D_0}{D}\left(1+\frac{-2\frac{dD}{dt}\eta^{UM}_{eff,0}\left(\frac{D_0}{D}\right)^2\frac{L_0}{s}}{2 \tau_{B,0}D_0}\right)\\
\label{eq:effective_stress1}
\end{align}

The next part of the derivation is exploitivng some relation that I wrote above ($\tau_{B,0}=2\eta^S_{eff}\dot{\varepsilon_c}$ and $\frac{dD}{dt}_c = \dot{\varepsilon_c}D_0$)

\begin{align}
   \tau = 2\eta^S_{eff,0}\dot{\varepsilon}_c\frac{ D_0}{D} &\left(1+\frac{-\frac{dD}{dt}\eta^{UM}_{eff,0}\left(\frac{D_0}{D}\right)^2\frac{L_0}{s}}{2\eta^S_{eff,0}\dot{\varepsilon}_c D_0}\right)\\
   &\left( 1-\frac{\Psi L_0 \alpha}{2 s}\left(\frac{D_0}{D}\right)^2 \frac{dD}{dt}\left(\frac{dD}{dt}_c\right)^{-1}\right)
\end{align}
then the equation for the effetive stress becomes
\begin{equation}
    \tau = \tau_{B,0} \frac{D_0}{D} \left( 1-\frac{\Psi L_0 \alpha}{2 s}\left(\frac{D_0}{D}\right)^2 \frac{dD}{dt}\left(\frac{dD}{dt}_c\right)^{-1}\right)
\end{equation}
which is properly not dimensionalized with $\tau_c = \tau_{B,0}$, and using additional characteristic value defined above: 
\begin{equation}
    \frac{\tau}{\tau_c} = \tau^{a}_{B} \left( 1-\frac{\Psi L_0 \alpha}{2 s}\left(\tau^{a}_B\right)^2 \frac{dD^{a}}{dt^{a}}\right)
    \label{eq:no_d_tau_eff}
\end{equation}

Additionally i can introduce the following term: 

\begin{equation}
    \gamma = \frac{L_0 \alpha}{2 s}
\end{equation}
Which represent adimensional group concerning the characteristic wavelegnth of deformation w.r.t the length of the slab and the scale of the convection. 
Yielding: 

\begin{equation}
    \frac{\tau}{\tau_c} = \tau^{a}_{B} \left( 1-\gamma \Psi \tau^{a}_B^2 \frac{dD^{a}}{dt^{a}}\right)
    \label{eq:no_d_tau_eff}
\end{equation}
or introducing $\Lambda= \gamma \Psi$: 

\begin{equation}
    \frac{\tau}{\tau_c} = \tau^{a}_{B} \left( 1-\Lambda \tau^{a}_B^2 \frac{dD^{a}}{dt^{a}}\right)
    \label{eq:no_d_tau_eff}
\end{equation}

\section{Final equation}

\begin{align}
    0 &= -D^{a} \left\{B_n^{a}  \left(\tau^{a}\right)^n \left[ 1+ \left( \frac{\tau^{a}}{\xi} \right)^{1-n} \right] \right\} -\left(\frac{dD^{a}}{dt^{a}}\right)\\
    &=-D^{a} \left\{B_n^{a}  \left(\tau^{a}_{B} \left( 1-\Lambda \tau^{a}_B^2 \frac{dD^{a}}{dt^{a}}\right)\right)^n \left[ 1+ \left( \frac{\tau^{a}_{B} \left( 1-\Lambda \tau^{a}_B^2 \frac{dD^{a}}{dt^{a}}\right)}{\xi} \right)^{1-n} \right] \right\} -{dD^{a}}{dt^{a}}
  \label{eq:Main_equation}
\end{align}
since we define $\xi = \frac{1}{\sqrt[n-1]{\Xi}}$ which is equivalent to $\xi=\Xi^{\frac{1}{1-n}}$. Then $\frac{1}{\xi}=\frac{1}{\Xi^{\frac{1}{1-n}}}$ which allows to extract $\frac{1}{\Xi}$
\begin{equation}
    0 =-D^{a}\left\{B_n^{a}\left(\tau^{a}_{B} \left( 1- \Lambda \tau^{a}_B^2 \frac{dD^{a}}{dt^{a}}\right)\right)^n \left[ 1+ \frac{1}{\Xi}\left( \tau^{a}_{B} \left( 1-\Lambda\tau^{a}_B^2 \frac{dD^{a}}{dt^{a}}\right)\right)^{1-n} \right] \right\} -\frac{dD^{a}}{dt^{a}}
  \label{eq:Main_equation}
\end{equation}

\begin{align}
    B_n^{a} &= \frac{B_n}{(\tau_{B,0})^{-n}\dot{\varepsilon}_c} && \tau^{a}_{B} &= \frac{\tau_{B}}{\tau_{B,0}} = \frac{D_0}{D}\\ 
    D^{a} &= \frac{D}{D_0}  && \frac{dD^{a}}{dt} &= \frac{\frac{dD}{dt}}{D_0 \dot{\varepsilon}_c} \\
    \Lambda &= \gamma \Psi && \gamma &= \frac{L_0\alpha}{s} \\ 
    \Psi &= \frac{\eta^{UM}_{eff,0}}{\eta^{S}_{eff,0}} && \Xi &= \frac{\eta^{S}_{d,0}}{\eta^{S}_{d,0}}
\end{align}

\section{Non Linear Upper mantle equations }

In the following part, I would like to start a derivation for integrating non linear upper mantle in the 0D numerical simulation. First the basic assumption is the following that the drag force acting on the slab is the same acting on the mantle. Second that the new form of the not dimensional equations is involving a $\Lambda(\tau)$. 

I keep the same definition of reference viscosity. Moreover, the same scaling between $\eta_{0,D}=\xi\eta_{0,D}$. 

At reference condition (i.e. $\tau = \tau_{0,B} = \tau_c$: 
\begin{align}
    \eta_{0,D} &= \xi \eta_{0,n}\\
    \eta_{0,n} &= \frac{1}{\xi} \eta_{0,D}
\end{align}
Which imply: 
\begin{align}
    B_n        &= \frac{\tau_{0,B}^{1-n}}{2\eta_{0,n}}\\
    B_n        &= \frac{\tau_{0,B}^{1-n}}{2\frac{1}{\xi}\eta_{0,D}}    
\end{align}
thus: 
\begin{align}
    \eta_{n} &= \frac{1}{2}\frac{1}{B_n}\tau^{1-n}\\
    \eta_{n} &= \frac{1}{2}\frac{1}{\frac{\tau_{0,B}^{1-n}}{2\frac{1}{\xi}\eta_{0,D}} }\tau^{1-n}\\
    \eta_{n} &= \frac{\frac{1}{\xi}\eta_{0,D}}{{\tau_{0,B}^{1-n}} }\tau^{1-n}\\
    \eta_{n} &= \frac{1}{\xi}\eta_{0,D} \left(\frac{\tau}{\tau_{0,B}}\right)^{1-n}
\end{align}
Reducing the effective viscosity of the upper mantle: 
\begin{align}
    \eta_{eff} &= \left(\frac{1}{\eta_D}+\frac{1}{\eta_n}\right)^{-1}\\
    \eta_{eff} &= \left(\frac{1}{\eta_{0,D}}+\frac{1}{\frac{1}{\xi}\eta_{0,D} \left(\frac{\tau}{\tau_{0,B}}\right)^{1-n}}\right)^{-1}\\
    &= \left( \frac{\frac{1}{\xi}\eta_{0,D}\left(\frac{\tau_D}{\tau_{0,B}}\right)^{1-n}}{\frac{1}{\xi}\left(\frac{\tau_D}{\tau_{0,B}}\right)^{1-n}+1}\right)
\end{align}

if $\tau_D = 0.0$  the effective should be zero. But if i do the following trick: 

\begin{align}
    \eta_{eff} &= \left( \frac{\frac{1}{\xi}\eta_{0,D}\left(\frac{\tau_D}{\tau_{0,B}}\right)^{1-n}}{\frac{1}{\xi}\left(\frac{\tau_D}{\tau_{0,B}}\right)^{1-n}\left(1+ \xi\left( \frac{\tau_D}{\tau_{0,B}}\right)^{n-1}\right)}\right)\\
    \eta_{eff} &= \left( \frac{\eta_{0,D}}{\left(1+ \xi\left( \frac{\tau_D}{\tau_{0,B}}\right)^{n-1}\right)}\right)
\end{align}
\begin{equation}
 \eta_{eff}
\begin{dcases}
    \eta_{0,D} & \text{if } \frac{\tau_D}{\tau_{B,0}}=0.0\\
   \frac{\eta_{D}}{1+\xi}& \text{if } \frac{\tau_D}{\tau_{B,0}}=1.0
\end{dcases}
\end{equation}
Therefore, I can express the effective viscosity of the upper mantle with this simple and catchy equation:
\begin{equation}
    \eta^{UM}_{eff}=\frac{1}{1+\xi\left(\frac{\tau_D}{\tau_{0,B}}\right)^{n-1}}\eta_{0,D}
\end{equation}
This implies that the effective stress can be defined as: 
\begin{align}
    \tau_{eff} &=\tau_{0,B} \frac{D_0}{D}\left(1+\Psi\gamma\frac{D_0^2}{D^2} \frac{dD^{a}}{dt}\right)\\
        &= \tau_{0,B} \frac{D_0}{D}\left(1+\frac{\frac{1}{1+\xi\left(\frac{\tau_D}{\tau_{0,B}}\right)^{n-1}}\eta^{UM}_{0,D}}{\eta^S_{0,n}}\gamma\frac{D_0^2}{D^2} \frac{dD^{a}}{dt}\right)\\
        &= \tau_{0,B} \frac{D_0}{D}\left(1+\frac{\Psi_0\gamma}{1+\xi\left(\frac{\tau_D}{\tau_{0,B}}\right)^{n-1}}\frac{D_0^2}{D^2} \frac{dD^{a}}{dt}\right)\\
        &= \tau_{0,B} \frac{D_0}{D}\left(1+\frac{\Lambda}{1+\xi\left(\frac{\tau_D}{\tau_{0,B}}\right)^{n-1}}\frac{D_0^2}{D^2} \frac{dD^{a}}{dt}\right)
\end{align}










\end{document}


