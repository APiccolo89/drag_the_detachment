\documentclass{article}
\usepackage[utf8]{inputenc}
\usepackage{amsmath}
\title{0D numerical solution}
\author{andreapiccolo89 }
\date{September 2022}

\begin{document}

\maketitle

\section{Introduction}

I take the work done during the coding friday and expanding a bit.

\begin{itemize}
    \item Setting up the equations starting from the definition of necking; 
    \begin{enumerate}
        \item Rheological consideration;
        \item Introduction of the main Forces ($F_B$ and $F_D$)
        \item Definition of $\tau_{B,0}$ and $\tau$; 
        \item Definition of the dimensional equation to solve, with the characteristic scaling
    \end{enumerate}
    
    \item Definition of the characteristic dimensions and additional relation
    
    \item Derivation of the $\tau$
    \begin{enumerate}
        \item Rearranging the equation associated with the force balance;
        \item Introduction of the following adimensional number: $\Lambda$,$\Psi$ and $\gamma$
        \item Definition of $\tau$ as  a function of $\frac{dD}{dt}$,$\tau_B$ and $\Lambda$,$\Psi$ and $\gamma$. 
    \end{enumerate}
    \item Final equation and introduction of the therminology
\end{itemize}


\section{Derivation 0D numerical equations}
\subsection{Main reference equation and introductory definitions}
We have:
\begin{equation}
    \dot{\varepsilon} = -\frac{1}{D} \frac{dD}{dt}
    \label{eq:eq1}
\end{equation}
and
\begin{equation}
    \dot{\varepsilon} = B_{n} \tau^n + B_{d} \tau
    \label{eq:rheology0}
\end{equation}

this can be expressed as:
\begin{equation}
    \dot{\varepsilon} = B_{n} \tau^n \left[1 + \frac{B_{d}}{B_{n}} \tau^{1-n}\right]
    \label{eq:rheology1}
\end{equation}

transition stress:
\begin{equation}
      \tau_t^{n-1} = \frac{B_{d}}{B_{n}}
\end{equation}
Put that in the equation above:
\begin{equation}
    \dot{\varepsilon} = B_{n} \tau^n \left[1 + \left(\frac{\tau}{\tau_t}\right)^{1-n}\right]
    \label{eq:rheology2}
\end{equation}

Now, I would like to add a further definition. In my numerical approach I imposed that $\eta_d$ is $\eta_d = \Xi \eta_n$ at $\tau_{B}$ (the buoyancy stress imposed by the stalled slab). Where $\Xi$ is the viscosity contrast between the two mechanisms at the reference condition (i.e. $\tau_B$). 

From this reference condition, I can derived $B_{n|d}$. Therefore, this strategy allows me to do the following: 

\begin{align}
B_d &= \frac{1}{2\eta_n\Xi} \\
B_n &= \frac{\tau_B^{1-n}}{2\eta_n\Xi}    
\end{align}

then: 
\begin{equation}
    \tau_t^{n-1} = \frac{\frac{1}{2\eta_n\Xi}}{\frac{\tau_B^{1-n}}{2\eta_n}}
    \label{eq:transition_stress0}
\end{equation}

doing some arrangments: 

\begin{equation}
     \sqrt[n-1]{\tau_t^{n-1}} =  \sqrt[n-1]{\frac{\tau_B^{n-1}}{\Xi}}
    \label{eq:transition_stress1}
\end{equation}
and by conveniently introducing $\xi = \Xi^\frac{1}{1-n}$

\begin{equation}
    \tau_t = \xi\tau_B
    \label{eq:transition_stress2}
\end{equation}


So, let's come back to the equation 1 and rearrange it to describe the problem that I want to solve: 

\begin{equation}
    \frac{dD}{dt} = -D \left\{ B_n \tau^n \left[ 1+ \left( \frac{\tau}{\xi \tau_B} \right)^{1-n} \right] \right\}
    \label{eq:equation_neck0}
\end{equation}

The stress in the necking region is given by:
\begin{equation}
    \tau = \frac{1}{2}\frac{F_B+F_D}{D}
\label{eq:effective_stress0}
\end{equation}

where: 
\begin{align}
    F_B &= \Delta \varrho g L_0 D_0 \\ 
    F_D &= -2\frac{dD}{dt}\eta^{UM}_{eff,0}\left(\frac{D_0}{D}\right)^2\frac{L_0 \alpha}{s}
    \label{eq:Drag_Force}
\end{align}

The equation from which I would like to start the derivation is the following

\begin{equation}
    0 = -D^{*} l_c \left\{ (B_n^{*} \tau_c^{-n}{t_c}^{-1})  \left(\tau^{*}\tau_c\right)^n \left[ 1+ \left( \frac{\left(\tau^{*}\tau_c\right)}{\xi (\tau^{*}_{B,0}\tau_c)} \right)^{1-n} \right] \right\} -\left(\frac{dD^{*}}{dt^{*}}\frac{l_c}{t_c}\right)
  \label{eq:Main_equation}
\end{equation}

Where $\tau^{*}= \tau/\tau_c$ represents the not dimensional effective stress. While $D^{*}$ represents the not dimensional thickness. 
\subsection{Characteristic value and additional relations}
Now let's start defining additional and important relation and what I believe are the most important characteristic length: 
\begin{align}
    l_c &= D_0 \\ 
    \tau_c &= \tau_{B,0}\\
    &= \frac{F_B}{2D_0} \\ 
    &= \frac{\Delta \varrho g L_0 D_0}{2 D_0} \\ 
    &= \frac{\Delta \varrho g L_0}{2}\\
    t_c &= \frac{1}{\dot{\varepsilon}_c} \\
    \dot{\varepsilon}_c &= \left\{ B_n \tau_c^n \left[ 1+ \left( \frac{\tau_c}{\xi \tau_B} \right)^{1-n} \right] \right\}\\
     \frac{dD}{dt}_{c} &= l_c \dot{\varepsilon}_c\\
     &= \frac{l_c}{t_c}
\end{align}

Additionally, it can be possible define an other quantity $\eta^S_{eff,0}$. This quantity is the effective viscosity of the slab at given reference condition (i.e. $\tau=\tau_{B,0}=\tau_c$). This quantity allows to define the $\tau_{B,0}$ in this alternative manner:

\begin{equation}
    \tau_c = \tau_{B,0} = 2\eta^S_{eff,0}\dot{\varepsilon}_c
    \label{eq:tau_B_0_eq0}
\end{equation}



Then, we can tackle the problem represent by $\tau$ and introducing other useful relation for the derivation: 

\begin{align}
    \tau_B &= \frac{\tau_{B,0}D_0}{D} [Pa] \\
    \tau_B^{*}&= \frac{D_0}{D} 
\end{align}
\subsection{Derivation of the effective stress}
The basic equation are set up. In the following part I derive a definition for $\tau$, the effective stress \eqref{eq:effective_stress0}, using the relations in \eqref{eq:Drag_Force}.

\begin{align}
\tau &= \frac{F_B}{2D}\left(1+\frac{F_D}{F_B}\right)\\
&= \frac{\tau_{B,0} D_0}{D}\left(1+\frac{-2\frac{dD}{dt}\eta^{UM}_{eff,0}\left(\frac{D_0}{D}\right)^2\frac{L_0}{s}}{2 \tau_{B,0}D_0}\right)\\
\label{eq:effective_stress1}
\end{align}

The next part of the derivation is exploitivng some relation that I wrote above ($\tau_{B,0}=2\eta^S_{eff}\dot{\varepsilon_c}$ and $\frac{dD}{dt}_c = \dot{\varepsilon_c}D_0$)

\begin{align}
   \tau = 2\eta^S_{eff,0}\dot{\varepsilon}_c\frac{ D_0}{D} &\left(1+\frac{-\frac{dD}{dt}\eta^{UM}_{eff,0}\left(\frac{D_0}{D}\right)^2\frac{L_0}{s}}{2\eta^S_{eff,0}\dot{\varepsilon}_c D_0}\right)\\
   &\left( 1-\frac{\Psi L_0 \alpha}{2 s}\left(\frac{D_0}{D}\right)^2 \frac{dD}{dt}\left(\frac{dD}{dt}_c\right)^{-1}\right)
\end{align}
then the equation for the effetive stress becomes
\begin{equation}
    \tau = \tau_{B,0} \frac{D_0}{D} \left( 1-\frac{\Psi L_0 \alpha}{2 s}\left(\frac{D_0}{D}\right)^2 \frac{dD}{dt}\left(\frac{dD}{dt}_c\right)^{-1}\right)
\end{equation}
which is properly not dimensionalized with $\tau_c = \tau_{B,0}$, and using additional characteristic value defined above: 

\begin{equation}
    \frac{\tau}{\tau_c} = \tau^{*}_{B} \left( 1-\frac{\Psi L_0 \alpha}{2s}\left({\tau^{*}_B}\right)^2 \frac{dD^{*}}{dt^{*}}\right)
    \label{eq:no_d_tau_eff}
\end{equation}

Additionally i can introduce the following term: 

\begin{equation}
    \gamma = \frac{L_0 \alpha}{2 s}
\end{equation}
Which represent adimensional group concerning the characteristic wavelegnth of deformation w.r.t the length of the slab and the scale of the convection. 
Yielding: 

\begin{equation}
    \frac{\tau}{\tau_c} = \tau^{*}_{B} \left( 1-\gamma \Psi {\tau^{*}_B}^2 \frac{dD^{*}}{dt^{*}}\right)
    \label{eq:no_d_tau_eff}
\end{equation}
or introducing $\Lambda= \gamma \Psi$: 

\begin{equation}
    \frac{\tau}{\tau_c} = \tau^{*}_{B} \left( 1-\Lambda {\tau^{*}_B}^2 \frac{dD^{*}}{dt^{*}}\right)
    \label{eq:no_d_tau_eff}
\end{equation}

\section{Final equation}

\begin{align}
    0 &= -D^{*} \left\{B_n^{*}  \left(\tau^{*}\right)^n \left[ 1+ \left( \frac{\tau^{*}}{\xi} \right)^{1-n} \right] \right\} -\left(\frac{dD^{*}}{dt^{*}}\right)\\
    &=-D^{*} \left\{B_n^{*}  \left(\tau^{*}_{B} \left( 1-\Lambda {\tau^{*}_B}^2 \frac{dD^{*}}{dt^{*}}\right)\right)^n \left[ 1+ \left( \frac{\tau^{*}_{B} \left( 1-\Lambda {\tau^{*}_B}^2 \frac{dD^{*}}{dt^{*}}\right)}{\xi} \right)^{1-n} \right] \right\} -{dD^{*}}{dt^{*}}
  \label{eq:Main_equation}
\end{align}
since we define $\xi = \frac{1}{\sqrt[n-1]{\Xi}}$ which is equivalent to $\xi=\Xi^{\frac{1}{1-n}}$. Then $\frac{1}{\xi}=\frac{1}{\Xi^{\frac{1}{1-n}}}$ which allows to extract $\frac{1}{\Xi}$
\begin{equation}
    0 =-D^{*}\left\{B_n^{*}\left(\tau^{*}_{B} \left( 1- \Lambda {\tau^{*}_B}^2 \frac{dD^{*}}{dt^{*}}\right)\right)^n \left[ 1+ \frac{1}{\Xi}\left( \tau^{*}_{B} \left( 1-\Lambda{\tau^{*}_B}^2 \frac{dD^{*}}{dt^{*}}\right)\right)^{1-n} \right] \right\} -\frac{dD^{*}}{dt^{*}}
  \label{eq:Main_equation}
\end{equation}

\begin{align}
    B_n^{*} &= \frac{B_n}{(\tau_{B,0})^{-n}\dot{\varepsilon}_c} && \tau^{*}_{B} &= \frac{\tau_{B}}{\tau_{B,0}} = \frac{D_0}{D}\\ 
    D^{*} &= \frac{D}{D_0}  && \frac{dD^{*}}{dt} &= \frac{\frac{dD}{dt}}{D_0 \dot{\varepsilon}_c} \\
    \Lambda &= \gamma \Psi && \gamma &= \frac{L_0\alpha}{s} \\ 
    \Psi &= \frac{\eta^{UM}_{eff,0}}{\eta^{S}_{eff,0}} && \Xi &= \frac{\eta^{S}_{d,0}}{\eta^{S}_{d,0}}
\end{align}




\end{document}